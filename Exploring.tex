% Options for packages loaded elsewhere
\PassOptionsToPackage{unicode}{hyperref}
\PassOptionsToPackage{hyphens}{url}
%
\documentclass[
]{article}
\usepackage{lmodern}
\usepackage{amsmath}
\usepackage{ifxetex,ifluatex}
\ifnum 0\ifxetex 1\fi\ifluatex 1\fi=0 % if pdftex
  \usepackage[T1]{fontenc}
  \usepackage[utf8]{inputenc}
  \usepackage{textcomp} % provide euro and other symbols
  \usepackage{amssymb}
\else % if luatex or xetex
  \usepackage{unicode-math}
  \defaultfontfeatures{Scale=MatchLowercase}
  \defaultfontfeatures[\rmfamily]{Ligatures=TeX,Scale=1}
\fi
% Use upquote if available, for straight quotes in verbatim environments
\IfFileExists{upquote.sty}{\usepackage{upquote}}{}
\IfFileExists{microtype.sty}{% use microtype if available
  \usepackage[]{microtype}
  \UseMicrotypeSet[protrusion]{basicmath} % disable protrusion for tt fonts
}{}
\makeatletter
\@ifundefined{KOMAClassName}{% if non-KOMA class
  \IfFileExists{parskip.sty}{%
    \usepackage{parskip}
  }{% else
    \setlength{\parindent}{0pt}
    \setlength{\parskip}{6pt plus 2pt minus 1pt}}
}{% if KOMA class
  \KOMAoptions{parskip=half}}
\makeatother
\usepackage{xcolor}
\IfFileExists{xurl.sty}{\usepackage{xurl}}{} % add URL line breaks if available
\IfFileExists{bookmark.sty}{\usepackage{bookmark}}{\usepackage{hyperref}}
\hypersetup{
  hidelinks,
  pdfcreator={LaTeX via pandoc}}
\urlstyle{same} % disable monospaced font for URLs
\usepackage[margin=1in]{geometry}
\usepackage{color}
\usepackage{fancyvrb}
\newcommand{\VerbBar}{|}
\newcommand{\VERB}{\Verb[commandchars=\\\{\}]}
\DefineVerbatimEnvironment{Highlighting}{Verbatim}{commandchars=\\\{\}}
% Add ',fontsize=\small' for more characters per line
\usepackage{framed}
\definecolor{shadecolor}{RGB}{248,248,248}
\newenvironment{Shaded}{\begin{snugshade}}{\end{snugshade}}
\newcommand{\AlertTok}[1]{\textcolor[rgb]{0.94,0.16,0.16}{#1}}
\newcommand{\AnnotationTok}[1]{\textcolor[rgb]{0.56,0.35,0.01}{\textbf{\textit{#1}}}}
\newcommand{\AttributeTok}[1]{\textcolor[rgb]{0.77,0.63,0.00}{#1}}
\newcommand{\BaseNTok}[1]{\textcolor[rgb]{0.00,0.00,0.81}{#1}}
\newcommand{\BuiltInTok}[1]{#1}
\newcommand{\CharTok}[1]{\textcolor[rgb]{0.31,0.60,0.02}{#1}}
\newcommand{\CommentTok}[1]{\textcolor[rgb]{0.56,0.35,0.01}{\textit{#1}}}
\newcommand{\CommentVarTok}[1]{\textcolor[rgb]{0.56,0.35,0.01}{\textbf{\textit{#1}}}}
\newcommand{\ConstantTok}[1]{\textcolor[rgb]{0.00,0.00,0.00}{#1}}
\newcommand{\ControlFlowTok}[1]{\textcolor[rgb]{0.13,0.29,0.53}{\textbf{#1}}}
\newcommand{\DataTypeTok}[1]{\textcolor[rgb]{0.13,0.29,0.53}{#1}}
\newcommand{\DecValTok}[1]{\textcolor[rgb]{0.00,0.00,0.81}{#1}}
\newcommand{\DocumentationTok}[1]{\textcolor[rgb]{0.56,0.35,0.01}{\textbf{\textit{#1}}}}
\newcommand{\ErrorTok}[1]{\textcolor[rgb]{0.64,0.00,0.00}{\textbf{#1}}}
\newcommand{\ExtensionTok}[1]{#1}
\newcommand{\FloatTok}[1]{\textcolor[rgb]{0.00,0.00,0.81}{#1}}
\newcommand{\FunctionTok}[1]{\textcolor[rgb]{0.00,0.00,0.00}{#1}}
\newcommand{\ImportTok}[1]{#1}
\newcommand{\InformationTok}[1]{\textcolor[rgb]{0.56,0.35,0.01}{\textbf{\textit{#1}}}}
\newcommand{\KeywordTok}[1]{\textcolor[rgb]{0.13,0.29,0.53}{\textbf{#1}}}
\newcommand{\NormalTok}[1]{#1}
\newcommand{\OperatorTok}[1]{\textcolor[rgb]{0.81,0.36,0.00}{\textbf{#1}}}
\newcommand{\OtherTok}[1]{\textcolor[rgb]{0.56,0.35,0.01}{#1}}
\newcommand{\PreprocessorTok}[1]{\textcolor[rgb]{0.56,0.35,0.01}{\textit{#1}}}
\newcommand{\RegionMarkerTok}[1]{#1}
\newcommand{\SpecialCharTok}[1]{\textcolor[rgb]{0.00,0.00,0.00}{#1}}
\newcommand{\SpecialStringTok}[1]{\textcolor[rgb]{0.31,0.60,0.02}{#1}}
\newcommand{\StringTok}[1]{\textcolor[rgb]{0.31,0.60,0.02}{#1}}
\newcommand{\VariableTok}[1]{\textcolor[rgb]{0.00,0.00,0.00}{#1}}
\newcommand{\VerbatimStringTok}[1]{\textcolor[rgb]{0.31,0.60,0.02}{#1}}
\newcommand{\WarningTok}[1]{\textcolor[rgb]{0.56,0.35,0.01}{\textbf{\textit{#1}}}}
\usepackage{graphicx}
\makeatletter
\def\maxwidth{\ifdim\Gin@nat@width>\linewidth\linewidth\else\Gin@nat@width\fi}
\def\maxheight{\ifdim\Gin@nat@height>\textheight\textheight\else\Gin@nat@height\fi}
\makeatother
% Scale images if necessary, so that they will not overflow the page
% margins by default, and it is still possible to overwrite the defaults
% using explicit options in \includegraphics[width, height, ...]{}
\setkeys{Gin}{width=\maxwidth,height=\maxheight,keepaspectratio}
% Set default figure placement to htbp
\makeatletter
\def\fps@figure{htbp}
\makeatother
\setlength{\emergencystretch}{3em} % prevent overfull lines
\providecommand{\tightlist}{%
  \setlength{\itemsep}{0pt}\setlength{\parskip}{0pt}}
\setcounter{secnumdepth}{-\maxdimen} % remove section numbering
\ifluatex
  \usepackage{selnolig}  % disable illegal ligatures
\fi

\author{}
\date{\vspace{-2.5em}}

\begin{document}

Required Packages and settings:

\begin{Shaded}
\begin{Highlighting}[]
\FunctionTok{library}\NormalTok{(tidyverse)}
\end{Highlighting}
\end{Shaded}

\begin{verbatim}
## -- Attaching packages --------------------------------------- tidyverse 1.3.0 --
\end{verbatim}

\begin{verbatim}
## v ggplot2 3.3.3     v purrr   0.3.4
## v tibble  3.0.6     v dplyr   1.0.3
## v tidyr   1.1.2     v stringr 1.4.0
## v readr   1.4.0     v forcats 0.5.1
\end{verbatim}

\begin{verbatim}
## -- Conflicts ------------------------------------------ tidyverse_conflicts() --
## x dplyr::filter() masks stats::filter()
## x dplyr::lag()    masks stats::lag()
\end{verbatim}

\begin{Shaded}
\begin{Highlighting}[]
\FunctionTok{library}\NormalTok{(data.table)}
\end{Highlighting}
\end{Shaded}

\begin{verbatim}
## 
## Attaching package: 'data.table'
\end{verbatim}

\begin{verbatim}
## The following objects are masked from 'package:dplyr':
## 
##     between, first, last
\end{verbatim}

\begin{verbatim}
## The following object is masked from 'package:purrr':
## 
##     transpose
\end{verbatim}

\begin{Shaded}
\begin{Highlighting}[]
\FunctionTok{library}\NormalTok{(ggplot2)}
\FunctionTok{library}\NormalTok{(dplyr)}
\FunctionTok{library}\NormalTok{(xtable)}
\end{Highlighting}
\end{Shaded}

Reading in Data, I'll assume that you have set your Working Directory,
and the file downloaded accordingly off the web.

\begin{Shaded}
\begin{Highlighting}[]
\NormalTok{StormData }\OtherTok{\textless{}{-}} \FunctionTok{read.csv}\NormalTok{(}\StringTok{"Data/repdata\_data\_StormData.csv.bz2"}
\NormalTok{                      , }\AttributeTok{header =} \ConstantTok{TRUE}
\NormalTok{                      , }\AttributeTok{sep =} \StringTok{","}\NormalTok{)}
\end{Highlighting}
\end{Shaded}

What it do?

\begin{Shaded}
\begin{Highlighting}[]
\FunctionTok{summary}\NormalTok{(StormData)}
\end{Highlighting}
\end{Shaded}

\begin{verbatim}
##     STATE__       BGN_DATE           BGN_TIME          TIME_ZONE        
##  Min.   : 1.0   Length:902297      Length:902297      Length:902297     
##  1st Qu.:19.0   Class :character   Class :character   Class :character  
##  Median :30.0   Mode  :character   Mode  :character   Mode  :character  
##  Mean   :31.2                                                           
##  3rd Qu.:45.0                                                           
##  Max.   :95.0                                                           
##                                                                         
##      COUNTY       COUNTYNAME           STATE              EVTYPE         
##  Min.   :  0.0   Length:902297      Length:902297      Length:902297     
##  1st Qu.: 31.0   Class :character   Class :character   Class :character  
##  Median : 75.0   Mode  :character   Mode  :character   Mode  :character  
##  Mean   :100.6                                                           
##  3rd Qu.:131.0                                                           
##  Max.   :873.0                                                           
##                                                                          
##    BGN_RANGE          BGN_AZI           BGN_LOCATI          END_DATE        
##  Min.   :   0.000   Length:902297      Length:902297      Length:902297     
##  1st Qu.:   0.000   Class :character   Class :character   Class :character  
##  Median :   0.000   Mode  :character   Mode  :character   Mode  :character  
##  Mean   :   1.484                                                           
##  3rd Qu.:   1.000                                                           
##  Max.   :3749.000                                                           
##                                                                             
##    END_TIME           COUNTY_END COUNTYENDN       END_RANGE       
##  Length:902297      Min.   :0    Mode:logical   Min.   :  0.0000  
##  Class :character   1st Qu.:0    NA's:902297    1st Qu.:  0.0000  
##  Mode  :character   Median :0                   Median :  0.0000  
##                     Mean   :0                   Mean   :  0.9862  
##                     3rd Qu.:0                   3rd Qu.:  0.0000  
##                     Max.   :0                   Max.   :925.0000  
##                                                                   
##    END_AZI           END_LOCATI            LENGTH              WIDTH         
##  Length:902297      Length:902297      Min.   :   0.0000   Min.   :   0.000  
##  Class :character   Class :character   1st Qu.:   0.0000   1st Qu.:   0.000  
##  Mode  :character   Mode  :character   Median :   0.0000   Median :   0.000  
##                                        Mean   :   0.2301   Mean   :   7.503  
##                                        3rd Qu.:   0.0000   3rd Qu.:   0.000  
##                                        Max.   :2315.0000   Max.   :4400.000  
##                                                                              
##        F               MAG            FATALITIES          INJURIES        
##  Min.   :0.0      Min.   :    0.0   Min.   :  0.0000   Min.   :   0.0000  
##  1st Qu.:0.0      1st Qu.:    0.0   1st Qu.:  0.0000   1st Qu.:   0.0000  
##  Median :1.0      Median :   50.0   Median :  0.0000   Median :   0.0000  
##  Mean   :0.9      Mean   :   46.9   Mean   :  0.0168   Mean   :   0.1557  
##  3rd Qu.:1.0      3rd Qu.:   75.0   3rd Qu.:  0.0000   3rd Qu.:   0.0000  
##  Max.   :5.0      Max.   :22000.0   Max.   :583.0000   Max.   :1700.0000  
##  NA's   :843563                                                           
##     PROPDMG         PROPDMGEXP           CROPDMG         CROPDMGEXP       
##  Min.   :   0.00   Length:902297      Min.   :  0.000   Length:902297     
##  1st Qu.:   0.00   Class :character   1st Qu.:  0.000   Class :character  
##  Median :   0.00   Mode  :character   Median :  0.000   Mode  :character  
##  Mean   :  12.06                      Mean   :  1.527                     
##  3rd Qu.:   0.50                      3rd Qu.:  0.000                     
##  Max.   :5000.00                      Max.   :990.000                     
##                                                                           
##      WFO             STATEOFFIC         ZONENAMES            LATITUDE   
##  Length:902297      Length:902297      Length:902297      Min.   :   0  
##  Class :character   Class :character   Class :character   1st Qu.:2802  
##  Mode  :character   Mode  :character   Mode  :character   Median :3540  
##                                                           Mean   :2875  
##                                                           3rd Qu.:4019  
##                                                           Max.   :9706  
##                                                           NA's   :47    
##    LONGITUDE        LATITUDE_E     LONGITUDE_       REMARKS         
##  Min.   :-14451   Min.   :   0   Min.   :-14455   Length:902297     
##  1st Qu.:  7247   1st Qu.:   0   1st Qu.:     0   Class :character  
##  Median :  8707   Median :   0   Median :     0   Mode  :character  
##  Mean   :  6940   Mean   :1452   Mean   :  3509                     
##  3rd Qu.:  9605   3rd Qu.:3549   3rd Qu.:  8735                     
##  Max.   : 17124   Max.   :9706   Max.   :106220                     
##                   NA's   :40                                        
##      REFNUM      
##  Min.   :     1  
##  1st Qu.:225575  
##  Median :451149  
##  Mean   :451149  
##  3rd Qu.:676723  
##  Max.   :902297  
## 
\end{verbatim}

\begin{Shaded}
\begin{Highlighting}[]
\FunctionTok{str}\NormalTok{(StormData)}
\end{Highlighting}
\end{Shaded}

\begin{verbatim}
## 'data.frame':    902297 obs. of  37 variables:
##  $ STATE__   : num  1 1 1 1 1 1 1 1 1 1 ...
##  $ BGN_DATE  : chr  "4/18/1950 0:00:00" "4/18/1950 0:00:00" "2/20/1951 0:00:00" "6/8/1951 0:00:00" ...
##  $ BGN_TIME  : chr  "0130" "0145" "1600" "0900" ...
##  $ TIME_ZONE : chr  "CST" "CST" "CST" "CST" ...
##  $ COUNTY    : num  97 3 57 89 43 77 9 123 125 57 ...
##  $ COUNTYNAME: chr  "MOBILE" "BALDWIN" "FAYETTE" "MADISON" ...
##  $ STATE     : chr  "AL" "AL" "AL" "AL" ...
##  $ EVTYPE    : chr  "TORNADO" "TORNADO" "TORNADO" "TORNADO" ...
##  $ BGN_RANGE : num  0 0 0 0 0 0 0 0 0 0 ...
##  $ BGN_AZI   : chr  "" "" "" "" ...
##  $ BGN_LOCATI: chr  "" "" "" "" ...
##  $ END_DATE  : chr  "" "" "" "" ...
##  $ END_TIME  : chr  "" "" "" "" ...
##  $ COUNTY_END: num  0 0 0 0 0 0 0 0 0 0 ...
##  $ COUNTYENDN: logi  NA NA NA NA NA NA ...
##  $ END_RANGE : num  0 0 0 0 0 0 0 0 0 0 ...
##  $ END_AZI   : chr  "" "" "" "" ...
##  $ END_LOCATI: chr  "" "" "" "" ...
##  $ LENGTH    : num  14 2 0.1 0 0 1.5 1.5 0 3.3 2.3 ...
##  $ WIDTH     : num  100 150 123 100 150 177 33 33 100 100 ...
##  $ F         : int  3 2 2 2 2 2 2 1 3 3 ...
##  $ MAG       : num  0 0 0 0 0 0 0 0 0 0 ...
##  $ FATALITIES: num  0 0 0 0 0 0 0 0 1 0 ...
##  $ INJURIES  : num  15 0 2 2 2 6 1 0 14 0 ...
##  $ PROPDMG   : num  25 2.5 25 2.5 2.5 2.5 2.5 2.5 25 25 ...
##  $ PROPDMGEXP: chr  "K" "K" "K" "K" ...
##  $ CROPDMG   : num  0 0 0 0 0 0 0 0 0 0 ...
##  $ CROPDMGEXP: chr  "" "" "" "" ...
##  $ WFO       : chr  "" "" "" "" ...
##  $ STATEOFFIC: chr  "" "" "" "" ...
##  $ ZONENAMES : chr  "" "" "" "" ...
##  $ LATITUDE  : num  3040 3042 3340 3458 3412 ...
##  $ LONGITUDE : num  8812 8755 8742 8626 8642 ...
##  $ LATITUDE_E: num  3051 0 0 0 0 ...
##  $ LONGITUDE_: num  8806 0 0 0 0 ...
##  $ REMARKS   : chr  "" "" "" "" ...
##  $ REFNUM    : num  1 2 3 4 5 6 7 8 9 10 ...
\end{verbatim}

\begin{Shaded}
\begin{Highlighting}[]
\FunctionTok{head}\NormalTok{(StormData)}
\end{Highlighting}
\end{Shaded}

\begin{verbatim}
##   STATE__           BGN_DATE BGN_TIME TIME_ZONE COUNTY COUNTYNAME STATE  EVTYPE
## 1       1  4/18/1950 0:00:00     0130       CST     97     MOBILE    AL TORNADO
## 2       1  4/18/1950 0:00:00     0145       CST      3    BALDWIN    AL TORNADO
## 3       1  2/20/1951 0:00:00     1600       CST     57    FAYETTE    AL TORNADO
## 4       1   6/8/1951 0:00:00     0900       CST     89    MADISON    AL TORNADO
## 5       1 11/15/1951 0:00:00     1500       CST     43    CULLMAN    AL TORNADO
## 6       1 11/15/1951 0:00:00     2000       CST     77 LAUDERDALE    AL TORNADO
##   BGN_RANGE BGN_AZI BGN_LOCATI END_DATE END_TIME COUNTY_END COUNTYENDN
## 1         0                                               0         NA
## 2         0                                               0         NA
## 3         0                                               0         NA
## 4         0                                               0         NA
## 5         0                                               0         NA
## 6         0                                               0         NA
##   END_RANGE END_AZI END_LOCATI LENGTH WIDTH F MAG FATALITIES INJURIES PROPDMG
## 1         0                      14.0   100 3   0          0       15    25.0
## 2         0                       2.0   150 2   0          0        0     2.5
## 3         0                       0.1   123 2   0          0        2    25.0
## 4         0                       0.0   100 2   0          0        2     2.5
## 5         0                       0.0   150 2   0          0        2     2.5
## 6         0                       1.5   177 2   0          0        6     2.5
##   PROPDMGEXP CROPDMG CROPDMGEXP WFO STATEOFFIC ZONENAMES LATITUDE LONGITUDE
## 1          K       0                                         3040      8812
## 2          K       0                                         3042      8755
## 3          K       0                                         3340      8742
## 4          K       0                                         3458      8626
## 5          K       0                                         3412      8642
## 6          K       0                                         3450      8748
##   LATITUDE_E LONGITUDE_ REMARKS REFNUM
## 1       3051       8806              1
## 2          0          0              2
## 3          0          0              3
## 4          0          0              4
## 5          0          0              5
## 6          0          0              6
\end{verbatim}

So 38 different columns (variables), and my suspicion is that we will
not end up utilizing all of them. For the sake of saving the CPU some
work, lets go ahead and condense this dataframe.

First we are ensuring that we are getting a non-zero number in atleast
one of the ``fatalities, injuries, property damage, and crop damage''
categories. After all, if they are not causing measurable damage they
should not show up in our findings.

Second, there is some cleaning we should do on the date types to make
sure that we are getting accurate data on the years that events are
taking place. The documentation alludes the distribution of weather
events are heavily skewed towards the later years, since the records are
spare. Even though there is a clear right skew to the data in regards to
year I believe this problem is not worth a subset.

Third, we need to look at the veracity of the ``EVTYPES'' column, to
make sure the events are properly recorded. According to the
documentation, there are only 47 event types. Lets see what we have:

\begin{Shaded}
\begin{Highlighting}[]
\NormalTok{CleanStormData }\OtherTok{\textless{}{-}} \FunctionTok{subset}\NormalTok{(StormData, (FATALITIES }\SpecialCharTok{\textgreater{}} \DecValTok{0} \SpecialCharTok{|}\NormalTok{ INJURIES }\SpecialCharTok{\textgreater{}} \DecValTok{0} \SpecialCharTok{|} 
\NormalTok{                                    PROPDMG }\SpecialCharTok{\textgreater{}} \DecValTok{0} \SpecialCharTok{|}\NormalTok{ CROPDMG }\SpecialCharTok{\textgreater{}} \DecValTok{0}\NormalTok{)}
\NormalTok{                                  , }\AttributeTok{select =}\FunctionTok{c}\NormalTok{(}\StringTok{"BGN\_DATE"}
\NormalTok{                                             ,}\StringTok{"STATE"}
\NormalTok{                                             ,}\StringTok{"EVTYPE"}
\NormalTok{                                             ,}\StringTok{"FATALITIES"}
\NormalTok{                                             ,}\StringTok{"INJURIES"}
\NormalTok{                                             ,}\StringTok{"PROPDMG"}
\NormalTok{                                             ,}\StringTok{"PROPDMGEXP"}
\NormalTok{                                             ,}\StringTok{"CROPDMG"}
\NormalTok{                                             ,}\StringTok{"CROPDMGEXP"}\NormalTok{))}

\NormalTok{CleanStormData}\SpecialCharTok{$}\NormalTok{year }\OtherTok{\textless{}{-}} \FunctionTok{as.numeric}\NormalTok{(}\FunctionTok{format}\NormalTok{(}
                                  \FunctionTok{as.Date}\NormalTok{(CleanStormData}\SpecialCharTok{$}\NormalTok{BGN\_DATE}
\NormalTok{                                        , }\AttributeTok{format =} \StringTok{"\%m/\%d/\%Y \%H:\%M:\%S"}\NormalTok{),}\StringTok{"\%Y"}\NormalTok{))}

\FunctionTok{hist}\NormalTok{(CleanStormData}\SpecialCharTok{$}\NormalTok{year, }\AttributeTok{breaks =} \DecValTok{10}\NormalTok{)}
\end{Highlighting}
\end{Shaded}

\includegraphics{Exploring_files/figure-latex/unnamed-chunk-4-1.pdf}

\begin{Shaded}
\begin{Highlighting}[]
\NormalTok{CleanStormData}\SpecialCharTok{$}\NormalTok{EVTYPE }\OtherTok{\textless{}{-}} \FunctionTok{toupper}\NormalTok{(CleanStormData}\SpecialCharTok{$}\NormalTok{EVTYPE)}
\end{Highlighting}
\end{Shaded}

Now, we're running into our second problem, the Property Damage and Crop
Damage columns. These documentation suggests that we get the following:

Symbol Magnitude ``B'' Billion ``M'' Million ``K'' Thousand

Simple enough. Lets see what we get.

\begin{Shaded}
\begin{Highlighting}[]
\FunctionTok{table}\NormalTok{(CleanStormData}\SpecialCharTok{$}\NormalTok{PROPDMGEXP) }
\end{Highlighting}
\end{Shaded}

\begin{verbatim}
## 
##             -      +      0      2      3      4      5      6      7      B 
##  11585      1      5    210      1      1      4     18      3      3     40 
##      h      H      K      m      M 
##      1      6 231428      7  11320
\end{verbatim}

\begin{Shaded}
\begin{Highlighting}[]
\FunctionTok{table}\NormalTok{(CleanStormData}\SpecialCharTok{$}\NormalTok{CROPDMGEXP)}
\end{Highlighting}
\end{Shaded}

\begin{verbatim}
## 
##             ?      0      B      k      K      m      M 
## 152664      6     17      7     21  99932      1   1985
\end{verbatim}

Alright, so there are a few other characters listed, including blanks

I'm making a few assumptions here:

\begin{verbatim}
1.) The capitalization is user generated and has no "real" difference. A "M" 
    will be treated the same as "m"

2.) The "H"/"h" character symbolizes a magnification of 100 

3.) The "-", "+", "0" "?" characters are missing value characters and will 
    have magnification of 1 (identity property)

4.) The numeric "2", "3", etc. yield the appropriate 10^X magnification, 
    where X = "2", X = "3" respectively. 
\end{verbatim}

With those assumptions in mind, here is the cleaning for the Economic
devastation

\begin{Shaded}
\begin{Highlighting}[]
\FunctionTok{head}\NormalTok{(}\FunctionTok{toupper}\NormalTok{(CleanStormData}\SpecialCharTok{$}\NormalTok{PROPDMGEXP),}\DecValTok{3}\NormalTok{)}
\end{Highlighting}
\end{Shaded}

\begin{verbatim}
## [1] "K" "K" "K"
\end{verbatim}

\begin{Shaded}
\begin{Highlighting}[]
\FunctionTok{head}\NormalTok{(}\FunctionTok{toupper}\NormalTok{(CleanStormData}\SpecialCharTok{$}\NormalTok{CROPDMGEXP),}\DecValTok{3}\NormalTok{)}
\end{Highlighting}
\end{Shaded}

\begin{verbatim}
## [1] "" "" ""
\end{verbatim}

\begin{Shaded}
\begin{Highlighting}[]
\NormalTok{Magnification }\OtherTok{\textless{}{-}} \ControlFlowTok{function}\NormalTok{(exp) \{}
\NormalTok{    exp }\OtherTok{\textless{}{-}} \FunctionTok{toupper}\NormalTok{(exp);}
    \ControlFlowTok{if}\NormalTok{ (exp }\SpecialCharTok{==} \StringTok{""}\NormalTok{)  }\FunctionTok{return}\NormalTok{ (}\DecValTok{10}\SpecialCharTok{\^{}}\DecValTok{0}\NormalTok{);}
    \ControlFlowTok{if}\NormalTok{ (exp }\SpecialCharTok{==} \StringTok{"{-}"}\NormalTok{) }\FunctionTok{return}\NormalTok{ (}\DecValTok{10}\SpecialCharTok{\^{}}\DecValTok{0}\NormalTok{);}
    \ControlFlowTok{if}\NormalTok{ (exp }\SpecialCharTok{==} \StringTok{"?"}\NormalTok{) }\FunctionTok{return}\NormalTok{ (}\DecValTok{10}\SpecialCharTok{\^{}}\DecValTok{0}\NormalTok{);}
    \ControlFlowTok{if}\NormalTok{ (exp }\SpecialCharTok{==} \StringTok{"+"}\NormalTok{) }\FunctionTok{return}\NormalTok{ (}\DecValTok{10}\SpecialCharTok{\^{}}\DecValTok{0}\NormalTok{);}
    \ControlFlowTok{if}\NormalTok{ (exp }\SpecialCharTok{==} \StringTok{"0"}\NormalTok{) }\FunctionTok{return}\NormalTok{ (}\DecValTok{10}\SpecialCharTok{\^{}}\DecValTok{0}\NormalTok{);}
    \ControlFlowTok{if}\NormalTok{ (exp }\SpecialCharTok{==} \StringTok{"1"}\NormalTok{) }\FunctionTok{return}\NormalTok{ (}\DecValTok{10}\SpecialCharTok{\^{}}\DecValTok{1}\NormalTok{);}
    \ControlFlowTok{if}\NormalTok{ (exp }\SpecialCharTok{==} \StringTok{"2"}\NormalTok{) }\FunctionTok{return}\NormalTok{ (}\DecValTok{10}\SpecialCharTok{\^{}}\DecValTok{2}\NormalTok{);}
    \ControlFlowTok{if}\NormalTok{ (exp }\SpecialCharTok{==} \StringTok{"3"}\NormalTok{) }\FunctionTok{return}\NormalTok{ (}\DecValTok{10}\SpecialCharTok{\^{}}\DecValTok{3}\NormalTok{);}
    \ControlFlowTok{if}\NormalTok{ (exp }\SpecialCharTok{==} \StringTok{"4"}\NormalTok{) }\FunctionTok{return}\NormalTok{ (}\DecValTok{10}\SpecialCharTok{\^{}}\DecValTok{4}\NormalTok{);}
    \ControlFlowTok{if}\NormalTok{ (exp }\SpecialCharTok{==} \StringTok{"5"}\NormalTok{) }\FunctionTok{return}\NormalTok{ (}\DecValTok{10}\SpecialCharTok{\^{}}\DecValTok{5}\NormalTok{);}
    \ControlFlowTok{if}\NormalTok{ (exp }\SpecialCharTok{==} \StringTok{"6"}\NormalTok{) }\FunctionTok{return}\NormalTok{ (}\DecValTok{10}\SpecialCharTok{\^{}}\DecValTok{6}\NormalTok{);}
    \ControlFlowTok{if}\NormalTok{ (exp }\SpecialCharTok{==} \StringTok{"7"}\NormalTok{) }\FunctionTok{return}\NormalTok{ (}\DecValTok{10}\SpecialCharTok{\^{}}\DecValTok{7}\NormalTok{);}
    \ControlFlowTok{if}\NormalTok{ (exp }\SpecialCharTok{==} \StringTok{"8"}\NormalTok{) }\FunctionTok{return}\NormalTok{ (}\DecValTok{10}\SpecialCharTok{\^{}}\DecValTok{8}\NormalTok{);}
    \ControlFlowTok{if}\NormalTok{ (exp }\SpecialCharTok{==} \StringTok{"9"}\NormalTok{) }\FunctionTok{return}\NormalTok{ (}\DecValTok{10}\SpecialCharTok{\^{}}\DecValTok{9}\NormalTok{);}
    \ControlFlowTok{if}\NormalTok{ (exp }\SpecialCharTok{==} \StringTok{"H"}\NormalTok{) }\FunctionTok{return}\NormalTok{ (}\DecValTok{10}\SpecialCharTok{\^{}}\DecValTok{2}\NormalTok{);}
    \ControlFlowTok{if}\NormalTok{ (exp }\SpecialCharTok{==} \StringTok{"K"}\NormalTok{) }\FunctionTok{return}\NormalTok{ (}\DecValTok{10}\SpecialCharTok{\^{}}\DecValTok{3}\NormalTok{);}
    \ControlFlowTok{if}\NormalTok{ (exp }\SpecialCharTok{==} \StringTok{"M"}\NormalTok{) }\FunctionTok{return}\NormalTok{ (}\DecValTok{10}\SpecialCharTok{\^{}}\DecValTok{6}\NormalTok{);}
    \ControlFlowTok{if}\NormalTok{ (exp }\SpecialCharTok{==} \StringTok{"B"}\NormalTok{) }\FunctionTok{return}\NormalTok{ (}\DecValTok{10}\SpecialCharTok{\^{}}\DecValTok{9}\NormalTok{);}
    \FunctionTok{return}\NormalTok{ (}\ConstantTok{NA}\NormalTok{);}
\NormalTok{\}}

\NormalTok{CleanStormData}\SpecialCharTok{$}\NormalTok{TotalPropCost }\OtherTok{\textless{}{-}} \FunctionTok{with}\NormalTok{(CleanStormData, }\FunctionTok{as.numeric}\NormalTok{(PROPDMG) }
                               \SpecialCharTok{*} \FunctionTok{sapply}\NormalTok{(PROPDMGEXP, Magnification))}\SpecialCharTok{/}\DecValTok{10}\SpecialCharTok{\^{}}\DecValTok{9}

\NormalTok{CleanStormData}\SpecialCharTok{$}\NormalTok{TotalCropCost }\OtherTok{\textless{}{-}} \FunctionTok{with}\NormalTok{(CleanStormData, }\FunctionTok{as.numeric}\NormalTok{(CROPDMG) }
                              \SpecialCharTok{*} \FunctionTok{sapply}\NormalTok{(CROPDMGEXP, Magnification))}\SpecialCharTok{/}\DecValTok{10}\SpecialCharTok{\^{}}\DecValTok{9}
\end{Highlighting}
\end{Shaded}

Our final step before actually plotting and visualizing our results is
making sure that our data is properly aggregated (by sum) and ordered.

\begin{Shaded}
\begin{Highlighting}[]
\CommentTok{\#Impact on Public Health}
\NormalTok{Fatalities }\OtherTok{\textless{}{-}} \FunctionTok{aggregate}\NormalTok{(}\AttributeTok{x =} \FunctionTok{list}\NormalTok{(}\AttributeTok{Impact =}\NormalTok{ CleanStormData}\SpecialCharTok{$}\NormalTok{FATALITIES)}
\NormalTok{                               , }\AttributeTok{by =} \FunctionTok{list}\NormalTok{(}\AttributeTok{EVENT\_TYPE =}\NormalTok{ CleanStormData}\SpecialCharTok{$}\NormalTok{EVTYPE)}
\NormalTok{                               , }\AttributeTok{FUN =}\NormalTok{ sum)}

\NormalTok{Fatalities }\OtherTok{\textless{}{-}}\NormalTok{ Fatalities[}\FunctionTok{order}\NormalTok{(Fatalities}\SpecialCharTok{$}\NormalTok{Impact, }
                               \AttributeTok{decreasing =} \ConstantTok{TRUE}\NormalTok{),]}

\CommentTok{\# Impact on the Economy }
\NormalTok{PropertyCost }\OtherTok{\textless{}{-}} \FunctionTok{aggregate}\NormalTok{(}\AttributeTok{x =} \FunctionTok{list}\NormalTok{(}\AttributeTok{Impact =}\NormalTok{ CleanStormData}\SpecialCharTok{$}\NormalTok{TotalPropCost)}
\NormalTok{                                 , }\AttributeTok{by =} \FunctionTok{list}\NormalTok{(}\AttributeTok{EVENT\_TYPE=}\NormalTok{ CleanStormData}\SpecialCharTok{$}\NormalTok{EVTYPE)}
\NormalTok{                                 , }\AttributeTok{FUN =}\NormalTok{ sum) }

\NormalTok{PropertyCost }\OtherTok{\textless{}{-}}\NormalTok{ PropertyCost[}\FunctionTok{order}\NormalTok{(PropertyCost}\SpecialCharTok{$}\NormalTok{Impact,}
                                   \AttributeTok{decreasing =} \ConstantTok{TRUE}\NormalTok{),]}

\NormalTok{CropCost }\OtherTok{\textless{}{-}} \FunctionTok{aggregate}\NormalTok{(}\AttributeTok{x =} \FunctionTok{list}\NormalTok{(}\AttributeTok{Impact =}\NormalTok{ CleanStormData}\SpecialCharTok{$}\NormalTok{TotalCropCost)}
\NormalTok{                              , }\AttributeTok{by =} \FunctionTok{list}\NormalTok{(}\AttributeTok{EVENT\_TYPE =}\NormalTok{ CleanStormData}\SpecialCharTok{$}\NormalTok{EVTYPE)}
\NormalTok{                              , }\AttributeTok{FUN =}\NormalTok{ sum) }

\NormalTok{CropCost }\OtherTok{\textless{}{-}}\NormalTok{ CropCost[}\FunctionTok{order}\NormalTok{(CropCost}\SpecialCharTok{$}\NormalTok{Impact, }
                           \AttributeTok{decreasing =} \ConstantTok{TRUE}\NormalTok{),]}
\end{Highlighting}
\end{Shaded}

At this point, we are finally ready to visualize our results!

\begin{Shaded}
\begin{Highlighting}[]
\FunctionTok{print}\NormalTok{(}\FunctionTok{xtable}\NormalTok{(}\FunctionTok{head}\NormalTok{(PropertyCost, }\DecValTok{10}\NormalTok{),}
             \AttributeTok{caption =} \StringTok{"Top 10 Weather Events Most Harmful to Property"}\NormalTok{),}
             \AttributeTok{caption.placement =} \StringTok{\textquotesingle{}top\textquotesingle{}}\NormalTok{,}
             \AttributeTok{type =} \StringTok{"html"}\NormalTok{,}
             \AttributeTok{include.rownames =} \ConstantTok{FALSE}\NormalTok{,}
             \AttributeTok{html.table.attributes=}\StringTok{\textquotesingle{}class="table{-}bordered", width="100\%"\textquotesingle{}}\NormalTok{)}
\end{Highlighting}
\end{Shaded}

\begin{verbatim}
## <!-- html table generated in R 4.0.3 by xtable 1.8-4 package -->
## <!-- Tue Feb  2 14:14:30 2021 -->
## <table class="table-bordered", width="100%">
## <caption align="top"> Top 10 Weather Events Most Harmful to Property </caption>
## <tr> <th> EVENT_TYPE </th> <th> Impact </th>  </tr>
##   <tr> <td> FLOOD </td> <td align="right"> 144.66 </td> </tr>
##   <tr> <td> HURRICANE/TYPHOON </td> <td align="right"> 69.31 </td> </tr>
##   <tr> <td> TORNADO </td> <td align="right"> 56.95 </td> </tr>
##   <tr> <td> STORM SURGE </td> <td align="right"> 43.32 </td> </tr>
##   <tr> <td> FLASH FLOOD </td> <td align="right"> 16.82 </td> </tr>
##   <tr> <td> HAIL </td> <td align="right"> 15.74 </td> </tr>
##   <tr> <td> HURRICANE </td> <td align="right"> 11.87 </td> </tr>
##   <tr> <td> TROPICAL STORM </td> <td align="right"> 7.70 </td> </tr>
##   <tr> <td> WINTER STORM </td> <td align="right"> 6.69 </td> </tr>
##   <tr> <td> HIGH WIND </td> <td align="right"> 5.27 </td> </tr>
##    </table>
\end{verbatim}

\begin{Shaded}
\begin{Highlighting}[]
\FunctionTok{print}\NormalTok{(}\FunctionTok{xtable}\NormalTok{(}\FunctionTok{head}\NormalTok{(CropCost, }\DecValTok{10}\NormalTok{),}
             \AttributeTok{caption =} \StringTok{"Top 10 Weather Events Most Harmful to Crops"}\NormalTok{),}
             \AttributeTok{caption.placement =} \StringTok{\textquotesingle{}top\textquotesingle{}}\NormalTok{,}
             \AttributeTok{type =} \StringTok{"html"}\NormalTok{,}
             \AttributeTok{include.rownames =} \ConstantTok{FALSE}\NormalTok{,}
             \AttributeTok{html.table.attributes=}\StringTok{\textquotesingle{}class="table{-}bordered", width="100\%"\textquotesingle{}}\NormalTok{)}
\end{Highlighting}
\end{Shaded}

\begin{verbatim}
## <!-- html table generated in R 4.0.3 by xtable 1.8-4 package -->
## <!-- Tue Feb  2 14:14:30 2021 -->
## <table class="table-bordered", width="100%">
## <caption align="top"> Top 10 Weather Events Most Harmful to Crops </caption>
## <tr> <th> EVENT_TYPE </th> <th> Impact </th>  </tr>
##   <tr> <td> DROUGHT </td> <td align="right"> 13.97 </td> </tr>
##   <tr> <td> FLOOD </td> <td align="right"> 5.66 </td> </tr>
##   <tr> <td> RIVER FLOOD </td> <td align="right"> 5.03 </td> </tr>
##   <tr> <td> ICE STORM </td> <td align="right"> 5.02 </td> </tr>
##   <tr> <td> HAIL </td> <td align="right"> 3.03 </td> </tr>
##   <tr> <td> HURRICANE </td> <td align="right"> 2.74 </td> </tr>
##   <tr> <td> HURRICANE/TYPHOON </td> <td align="right"> 2.61 </td> </tr>
##   <tr> <td> FLASH FLOOD </td> <td align="right"> 1.42 </td> </tr>
##   <tr> <td> EXTREME COLD </td> <td align="right"> 1.31 </td> </tr>
##   <tr> <td> FROST/FREEZE </td> <td align="right"> 1.09 </td> </tr>
##    </table>
\end{verbatim}

\begin{Shaded}
\begin{Highlighting}[]
\FunctionTok{print}\NormalTok{(}\FunctionTok{xtable}\NormalTok{(}\FunctionTok{head}\NormalTok{(Fatalities, }\DecValTok{10}\NormalTok{),}
             \AttributeTok{caption =} \StringTok{"Top 10 Deadliest Weather Events"}\NormalTok{),}
             \AttributeTok{caption.placement =} \StringTok{\textquotesingle{}top\textquotesingle{}}\NormalTok{,}
             \AttributeTok{type =} \StringTok{"html"}\NormalTok{,}
             \AttributeTok{include.rownames =} \ConstantTok{FALSE}\NormalTok{,}
             \AttributeTok{html.table.attributes=}\StringTok{\textquotesingle{}class="table{-}bordered", width="100\%"\textquotesingle{}}\NormalTok{)}
\end{Highlighting}
\end{Shaded}

\begin{verbatim}
## <!-- html table generated in R 4.0.3 by xtable 1.8-4 package -->
## <!-- Tue Feb  2 14:14:30 2021 -->
## <table class="table-bordered", width="100%">
## <caption align="top"> Top 10 Deadliest Weather Events </caption>
## <tr> <th> EVENT_TYPE </th> <th> Impact </th>  </tr>
##   <tr> <td> TORNADO </td> <td align="right"> 5633.00 </td> </tr>
##   <tr> <td> EXCESSIVE HEAT </td> <td align="right"> 1903.00 </td> </tr>
##   <tr> <td> FLASH FLOOD </td> <td align="right"> 978.00 </td> </tr>
##   <tr> <td> HEAT </td> <td align="right"> 937.00 </td> </tr>
##   <tr> <td> LIGHTNING </td> <td align="right"> 816.00 </td> </tr>
##   <tr> <td> TSTM WIND </td> <td align="right"> 504.00 </td> </tr>
##   <tr> <td> FLOOD </td> <td align="right"> 470.00 </td> </tr>
##   <tr> <td> RIP CURRENT </td> <td align="right"> 368.00 </td> </tr>
##   <tr> <td> HIGH WIND </td> <td align="right"> 248.00 </td> </tr>
##   <tr> <td> AVALANCHE </td> <td align="right"> 224.00 </td> </tr>
##    </table>
\end{verbatim}

First we are going to visualize the most damaging events to the US
economy, the property and weather related costs. After picking the ``top
10'' most dangerous events in each category. Floods and droughts are
bad.

\begin{Shaded}
\begin{Highlighting}[]
\NormalTok{CropCostPlot }\OtherTok{\textless{}{-}} \FunctionTok{ggplot}\NormalTok{(}\FunctionTok{head}\NormalTok{(CropCost,}\DecValTok{10}\NormalTok{)}
\NormalTok{                      , }\FunctionTok{aes}\NormalTok{(}\AttributeTok{x =}\NormalTok{ EVENT\_TYPE}
\NormalTok{                          , }\AttributeTok{y =}\NormalTok{ Impact)) }\SpecialCharTok{+}
                \FunctionTok{geom\_bar}\NormalTok{(}\AttributeTok{stat =} \StringTok{"Identity"}\NormalTok{) }\SpecialCharTok{+} 
                \FunctionTok{coord\_flip}\NormalTok{() }\SpecialCharTok{+}
                \FunctionTok{xlab}\NormalTok{(}\StringTok{"Event Type"}\NormalTok{) }\SpecialCharTok{+} 
                \FunctionTok{ylab}\NormalTok{(}\StringTok{"Total Crop Damage in USD (Billions)"}\NormalTok{) }\SpecialCharTok{+} 
                \FunctionTok{theme}\NormalTok{(}\AttributeTok{plot.title =} \FunctionTok{element\_text}\NormalTok{(}\AttributeTok{size =} \DecValTok{14}
\NormalTok{                                                , }\AttributeTok{hjust =} \FloatTok{0.5}\NormalTok{)) }\SpecialCharTok{+} 
                \FunctionTok{ggtitle}\NormalTok{(}\StringTok{"Top 10 Weather Events Most Impactful to Crops"}\NormalTok{)}

\NormalTok{PropertyCostPlot }\OtherTok{\textless{}{-}} \FunctionTok{ggplot}\NormalTok{(}\FunctionTok{head}\NormalTok{(PropertyCost,}\DecValTok{10}\NormalTok{)}
\NormalTok{                      , }\FunctionTok{aes}\NormalTok{(}\AttributeTok{x =}\NormalTok{ EVENT\_TYPE}
\NormalTok{                          , }\AttributeTok{y =}\NormalTok{ Impact)) }\SpecialCharTok{+}
                \FunctionTok{geom\_bar}\NormalTok{(}\AttributeTok{stat =} \StringTok{"Identity"}\NormalTok{) }\SpecialCharTok{+} 
                \FunctionTok{coord\_flip}\NormalTok{() }\SpecialCharTok{+}
                \FunctionTok{xlab}\NormalTok{(}\StringTok{"Event Type"}\NormalTok{) }\SpecialCharTok{+} 
                \FunctionTok{ylab}\NormalTok{(}\StringTok{"Total Crop Damage in USD (Billions)"}\NormalTok{) }\SpecialCharTok{+} 
                \FunctionTok{theme}\NormalTok{(}\AttributeTok{plot.title =} \FunctionTok{element\_text}\NormalTok{(}\AttributeTok{size =} \DecValTok{14}
\NormalTok{                                                , }\AttributeTok{hjust =} \FloatTok{0.5}\NormalTok{)) }\SpecialCharTok{+} 
                \FunctionTok{ggtitle}\NormalTok{(}\StringTok{"Top 10 Weather Events Most Impactful to Property"}\NormalTok{)}

\FunctionTok{print}\NormalTok{(CropCostPlot)}
\end{Highlighting}
\end{Shaded}

\includegraphics{Exploring_files/figure-latex/unnamed-chunk-9-1.pdf}

\begin{Shaded}
\begin{Highlighting}[]
\FunctionTok{print}\NormalTok{(PropertyCostPlot)}
\end{Highlighting}
\end{Shaded}

\includegraphics{Exploring_files/figure-latex/unnamed-chunk-9-2.pdf}

\begin{Shaded}
\begin{Highlighting}[]
\NormalTok{FatalitiesPlot }\OtherTok{\textless{}{-}} \FunctionTok{ggplot}\NormalTok{(}\FunctionTok{head}\NormalTok{(Fatalities,}\DecValTok{10}\NormalTok{)}
\NormalTok{                      , }\FunctionTok{aes}\NormalTok{(}\AttributeTok{x =}\NormalTok{ EVENT\_TYPE}
\NormalTok{                          , }\AttributeTok{y =}\NormalTok{ Impact)) }\SpecialCharTok{+}
                \FunctionTok{geom\_bar}\NormalTok{(}\AttributeTok{stat =} \StringTok{"Identity"}\NormalTok{) }\SpecialCharTok{+} 
                \FunctionTok{coord\_flip}\NormalTok{() }\SpecialCharTok{+}
                \FunctionTok{xlab}\NormalTok{(}\StringTok{"Event Type"}\NormalTok{) }\SpecialCharTok{+} 
                \FunctionTok{ylab}\NormalTok{(}\StringTok{"Aggregated Deaths"}\NormalTok{) }\SpecialCharTok{+} 
                \FunctionTok{theme}\NormalTok{(}\AttributeTok{plot.title =} \FunctionTok{element\_text}\NormalTok{(}\AttributeTok{size =} \DecValTok{14}
\NormalTok{                                                , }\AttributeTok{hjust =} \FloatTok{0.5}\NormalTok{)) }\SpecialCharTok{+} 
                \FunctionTok{ggtitle}\NormalTok{(}\StringTok{"Top 10 Deadliest Weather Events"}\NormalTok{)}


\FunctionTok{print}\NormalTok{(FatalitiesPlot)}
\end{Highlighting}
\end{Shaded}

\includegraphics{Exploring_files/figure-latex/unnamed-chunk-10-1.pdf}
Who Knew? Tornadoes = Bad

\end{document}
